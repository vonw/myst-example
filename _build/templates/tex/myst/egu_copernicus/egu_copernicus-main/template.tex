%% Copernicus Publications Manuscript Preparation Template for LaTeX Submissions
%% ---------------------------------
%% This template should be used for copernicus.cls
%% The class file and some style files are bundled in the Copernicus Latex Package, which can be downloaded from the different journal webpages.
%% For further assistance please contact Copernicus Publications at: production@copernicus.org
%% https://publications.copernicus.org/for_authors/manuscript_preparation.html


%% Please use the following documentclass and journal abbreviations for preprints and final revised papers.

%% 2-column papers and preprints
\documentclass[[-options.journal_name-], [# if options.proof #]proof[# else #]manuscript[# endif #]]{copernicus}


% Curvenote Pass Options Section
\PassOptionsToPackage{normalem}{ulem}
\PassOptionsToPackage{utf8}{inputenc}

% Curvenote Use Package Section
% base
\usepackage{inputenc}
\usepackage{url}
\usepackage{graphicx}
\usepackage{adjustbox}
\usepackage{amssymb}
\usepackage{amsfonts}
\usepackage{amsmath}
\usepackage{nicefrac}
\usepackage{microtype}
\usepackage{hyperref}
\usepackage{ulem}
\usepackage{enumitem}
\usepackage{float}
\usepackage{datetime}
\usepackage{xkeyval}
\usepackage{framed}
\usepackage{pdflscape}
\usepackage{booktabs}
\usepackage{longtable}
% template
\usepackage{fancyvrb}

[-IMPORTS-]
% colors for hyperlinks
\hypersetup{colorlinks=true, allcolors=blue}
\hypersetup{
pdftitle={\@title},
pdfsubject={[-doc.description-]},
pdfauthor={\@author},
pdfkeywords={[-doc.keywords-]},
addtopdfcreator={Written in Curvenote}
}

\begin{document}

\title{[-doc.title-]}

[# for author in doc.authors #]
\Author[[-author.index-]]{[-author.given_name-]}{[-author.surname-]}
[# endfor #]

[# for author in doc.authors #]
\affil[[-author.index-]]{
[#- for affiliation in author.affiliations #]
[-affiliation.value.name-][# if not loop.last #], [# endif #]
[# endfor #]}
[# endfor #]

%% The [] brackets identify the author with the corresponding affiliation. 1, 2, 3, etc. should be inserted.

%% If an author is deceased, please mark the respective author name(s) with a dagger, e.g. "\Author[2,$\dag$]{Anton}{Smith}", and add a further "\affil[$\dag$]{deceased, 1 July 2019}".

%% If authors contributed equally, please mark the respective author names with an asterisk, e.g. "\Author[2,*]{Anton}{Smith}" and "\Author[3,*]{Bradley}{Miller}" and add a further affiliation: "\affil[*]{These authors contributed equally to this work.}".

[# for author in doc.authors #]
[# if author.corresponding #]
\correspondence{[-author.name-] ([-author.email-])}
[# endif #]
[# endfor #]

\runningtitle{[-doc.short_title-]}

\runningauthor{[-options.running_author-]}

%% These dates will be inserted by Copernicus Publications during the typesetting process.
\received{}
\pubdiscuss{} %% only important for two-stage journals
\revised{}
\accepted{}
\published{}

\firstpage{1}

\maketitle

\begin{abstract}
[-parts.abstract-]
\end{abstract}


\copyrightstatement{[-options.copyright-]} %% This section is optional and can be used for copyright transfers.

[-CONTENT-]

%% The following commands are for the statements about the availability of data sets and/or software code corresponding to the manuscript.
%% It is strongly recommended to make use of these sections in case data sets and/or software code have been part of your research the article is based on.

[# if parts.code_availability #]
\codeavailability{[-parts.code_availability-]} %% use this section when having only software code available
[# endif #]

[# if parts.data_availability #]
\dataavailability{[-parts.data_availability-]} %% use this section when having only data sets available
[# endif #]

[# if parts.code_data_availability #]
\codedataavailability{[-parts.code_data_availability-]} %% use this section when having data sets and software code available
[# endif #]

[# if parts.sample_availability #]
\sampleavailability{[-parts.sample_availability-]} %% use this section when having geoscientific samples available
[# endif #]

[# if parts.video_supplement #]
\videosupplement{[-parts.video_supplement-]} %% use this section when having video supplements available
[# endif #]

[# if parts.appendix #]
\appendix
[-parts.appendix-]
\noappendix
[# endif #]

%% Regarding figures and tables in appendices, the following two options are possible depending on your general handling of figures and tables in the manuscript environment:

%% Option 1: If you sorted all figures and tables into the sections of the text, please also sort the appendix figures and appendix tables into the respective appendix sections.
%% They will be correctly named automatically.

%% Option 2: If you put all figures after the reference list, please insert appendix tables and figures after the normal tables and figures.
%% To rename them correctly to A1, A2, etc., please add the following commands in front of them:

[# if parts.appendix_figures #]
\appendixfigures  %% needs to be added in front of appendix figures
%% Please add \clearpage between each table and/or figure. Further guidelines on figures and tables can be found below.
[-parts.appendix_figures-]
[# endif #]

[# if parts.appendix_tables #]
\appendixtables   %% needs to be added in front of appendix tables
%% Please add \clearpage between each table and/or figure. Further guidelines on figures and tables can be found below.
[-parts.appendix_tables-]
[# endif #]

[# if parts.author_contribution #]
\authorcontribution{[-parts.author_contribution-]} %% this section is mandatory
[# endif #]

[# if parts.competing_interests #]
\competinginterests{[-parts.competing_interests-]} %% this section is mandatory even if you declare that no competing interests are present
[# endif #]

[# if parts.disclaimer #]
\disclaimer{[-parts.disclaimer-]} %% optional section
[# endif #]

[# if parts.acknowledgments #]
\acknowledgments{[-parts.acknowledgments-]}
[# endif #]


%% REFERENCES
\bibliographystyle{copernicus}
\bibliography{main}

\end{document}
